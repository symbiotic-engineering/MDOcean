% +------------------------------------+
% |   Generated by www.docx2latex.com  |
% |   Version: 2.0.0                   |
% |  Modified by Kelley Ruehl          |
% +------------------------------------+

\documentclass[10pt,twoside]{article}
\usepackage{amsmath}
\usepackage{framed}
\usepackage{adjustbox}
\usepackage{fancyhdr}
\usepackage{float}
\usepackage[T1]{fontenc}
\usepackage{graphicx}
\usepackage[utf8]{inputenc}
\usepackage{multicol}
\usepackage{multirow}
\usepackage{txfonts}
\usepackage[svgnames]{xcolor}
\usepackage{titlesec}
\usepackage{wrapfig}
\usepackage{caption}
\usepackage{lipsum}
\usepackage[backend=biber, sorting=none]{biblatex}
\addbibresource{references.bib}
\usepackage[paperheight=27.94cm,paperwidth=21.59cm,left=2.54cm,right=2.54cm,top=2.54cm,bottom=2.54cm]{geometry}
\usepackage{hyperref}

\titleformat*{\section}{\normalsize\bfseries}
\titleformat*{\subsection}{\normalsize\bfseries}
\titleformat*{\subsubsection}{\normalsize\bfseries}
%%%%%%%%%%%%%%%%%%%%%%%%%%%%%%%%%%%%%%%%%%%%%%%%%%%%%%%%%%%%%%%%%%%%%%%%%%%%%%%%%%%%%%%%%%%%%%%%%%%%%%%%
% Header
%%%%%%%%%%%%%%%%%%%%%%%%%%%%%%%%%%%%%%%%%%%%%%%%%%%%%%%%%%%%%%%%%%%%%%%%%%%%%%%%%%%%%%%%%%%%%%%%%%%%%%%%
%\setlength\parindent{0pt}
\renewcommand{\arraystretch}{1.3}\pagestyle{fancy}
\fancyhf{}
\lhead{\thepage}
\chead{\textit{McCabe $\vert$ Proceedings of UMERC+OREC 2025}}

%%%%%%%%%%%%%%%%%%%%%%%%%%%%%%%%%%%%%%%%%%%%%%%%%%%%%%%%%%%%%%%%%%%%%%%%%%%%%%%%%%%%%%%%%%%%%%%%%%%%%%%%
% Title
%%%%%%%%%%%%%%%%%%%%%%%%%%%%%%%%%%%%%%%%%%%%%%%%%%%%%%%%%%%%%%%%%%%%%%%%%%%%%%%%%%%%%%%%%%%%%%%%%%%%%%%%
\title{WEC optimization to maximize grid economic value and avoided emissions}
\author{Overleaf}
\date{\today}

\begin{document}\thispagestyle{empty} 

\begin{center}
    \begin{figure}[h]
        \centering
        \includegraphics[width = 0.2\textwidth]{./logo.jpg} \\
    \end{figure}
    \vspace{-\baselineskip}
    \huge
    UMERC+OREC \\
    2025 Conference \\
    \vspace{.5\baselineskip}
    \large
    \textit{12-14 August | Corvallis, OR USA} \\
    \vspace{1\baselineskip}
       
    \Large
    WEC optimization to maximize grid economic value and avoided emissions
    \vspace{.5\baselineskip}
    
    \large
    Rebecca McCabe\textsuperscript{a}\footnote{$\ast$ Corresponding author. E-mail address:  rgm222@cornell.edu \\ },
    Madison Dietrich\textsuperscript{a},
    Jiarui Yang\textsuperscript{a}, \\
    Anthony Long\textsuperscript{a},
    Khai Xin Kuan\textsuperscript{b},
    Leah Buccino\textsuperscript{a},
    Alan Liu\textsuperscript{c,d},
    Maha Haji\textsuperscript{a,e}
    
    
    \small
    \begin{center}
        \textit{\textsuperscript{a}Sibley School of Mechanical and Aerospace Engineering, Cornell University, 124 Hoy Road, Ithaca, NY 14853, USA} \\
        \textit{\textsuperscript{b}Cornell University Department of Information Science, 236 Gates Hall, Ithaca NY 14853, USA} \\
        \textit{\textsuperscript{c}Cornell University Department of Economics, 109 Tower Road, Ithaca, NY 14853, USA} \\
        \textit{\textsuperscript{d}Cornell University Department of Psychology, 116 Reservoir Ave, Ithaca, NY 14853, USA} \\
        \textit{\textsuperscript{e}Department of Systems Engineering, Cornell University, 136 Hoy Road, Ithaca, NY 14853, USA}
    \end{center}
    \normalsize
\end{center}         
\vspace{-\baselineskip}\noindent\rule{\textwidth}{0.4pt}\vspace{-\baselineskip}
%%%%%%%%%%%%%%%%%%%%%%%%%%%%%%%%%%%%%%%%%%%%%%%%%%%%%%%%%%%%%%%%%%%%%%%%%%%%%%%%%%%%%%%%%%%%%%%%%%%%%%%%
\section*{Abstract}
%%%%%%%%%%%%%%%%%%%%%%%%%%%%%%%%%%%%%%%%%%%%%%%%%%%%%%%%%%%%%%%%%%%%%%%%%%%%%%%%%%%%%%%%%%%%%%%%%%%%%%%%
Wave energy converter (WEC) design optimization has traditionally focused on minimizing the Levelized Cost of Energy (LCOE) or similar proxies.
However, this approach overlooks the reality of energy system planning, where capacity installation decisions are made to minimize total grid cost.
Grid system cost does not necessarily align with LCOE due to the complex temporal and spatial relationship between energy generation and demand.
Additionally, conventional WEC optimization neglects broader climate and electrification goals, where the reduction of lifetime equivalent CO\textsubscript{2} emissions is the key metric.
To bridge this gap, the authors previously proposed a system-level techno-economic and environmental WEC optimization framework that integrates capacity expansion modeling (CEM) and life cycle analysis (LCA) into the design objective.
This approach provides a more comprehensive assessment of wave energy's net value proposition beyond conventional cost metrics.
In this work, we implement this methodology in a new open-source multidisciplinary design optimization framework.
Our implementation leverages the GenX CEM, the PowerGenome energy data interface, the Idemat LCA dataset, and the MDOcean WEC model.
A surrogate model of the CEM reduces computation time compared to the naive CEM-in-the-loop approach.
We present preliminary optimization results for the Reference Model 3 (RM3) WEC, demonstrating the impact of optimizing for new value-driven economic and environmental system metrics compared to the standard LCOE.
For the scope of emissions controllable in the early design phase, the environmental objective aligns well with the economic one, indicating that design-for-environment techniques may not be relevant until later in the design process.
Meanwhile, the grid system economic objective is hypothesized to favor larger WECs in scenarios with winter generation deficits, such as the U.S. northeast with electrified heat loads, where the value of capturing energetic winter sea states outweighs the cost of surviving them.
On the other hand, systems with storage constraints should favor smaller WECs, where the avoided storage cost due to more consistent power generation outweighs the penalty in absorbed power.
Finally, we discuss the broader implications of these findings for future WEC design optimization priorities.
 \hfill 

\vspace{.5\baselineskip}
\textit{Keywords:} techno-economic model, environmental impact assessment, energy grid integration, multidisciplinary design optimization, life cycle analysis, capacity expansion model

\noindent\rule{\textwidth}{0.4pt}

%%%%%%%%%%%%%%%%%%%%%%%%%%%%%%%%%%%%%%%%%%%%%%%%%%%%%%%%%%%%%%%%%%%%%%%%%%%%%%%%%%%%%%%%%%%%%%%%%%%%%%%%
\section{Introduction}
%%%%%%%%%%%%%%%%%%%%%%%%%%%%%%%%%%%%%%%%%%%%%%%%%%%%%%%%%%%%%%%%%%%%%%%%%%%%%%%%%%%%%%%%%%%%%%%%%%%%%%%%
Advances in wave energy converter (WEC) multidisciplinary design optimization have reduced the Levelized Cost of Energy (LCOE) X-fold while incorporating realistic constraints such as structural survival, generator force limits, and body amplitude limits \cite{mccabe_mdocean_2024}.
However, LCOE does not fully capture the value of WECs.
Alternative metrics include net present value and payback period for non-grid Powering the Blue Economy contexts (cite Scott), and total grid system cost and lifetime equivalent CO\textsubscript{2} emissions in energy system planning and climate change mitigation.
Prior studies of WEC grid integration reveal benefits including reduced curtailment, increased capacity factor, and improved grid reliability due to seasonal complementarity with solar, proximity to coastal load, and wave resource consistency (cite WEC CEM/EDs and WEC grid value lit).
Considering energy system factors in the early design phase can steer WEC development to fully leverage this potential value and maximize climate benefit.

Recent studies for other technologies incorporate design considerations into energy system optimization (ie nuclear and long duration energy storage, cite Jesse Jenkins) and energy system considerations into design optimization (ie NVOE metric for offshore wind, cite WISDEM).
This study is the first to apply integrated design and energy system optimization to WECs, and also a methodological improvement over similar studies for other technologies. 
Specifically, the use of a surrogate capacity expansion model within a nonlinear design optimization more fully captures design-grid coupling while maintaining computational efficiency, a technique which the authors conceptually introduce in \cite{mccabe_system_2023} and here implement via dynamic system parameterization.
A second-order fit to a point absorber's nonlinear frequency response is found to be valid at frequencies at and below resonance.

   % -  prior work showing that CEM is more important than economic dispatch?

\section{Methodology}
\subsection{Optimization Problem Formulation}

    -  Design variables (quickly summarize MDOcean)
    -  Objectives: NVOE and net eco value of energy
    -  when not economically viable, avoid flatness by changing obj to margin to viability
    -  constraints (quickly summarize MDOcean)
    -  Algorithm: Summarize SQP, epsilon constraint and refer to MDOcean


\lipsum[1]

\subsection{Modeling}
\subsubsection{Overall Model Structure}

\begin{wrapfigure}[16]{r}{0.62\textwidth}
    \centering
    \includegraphics[width=.6\textwidth]{figures/out/xdsm_grid.pdf}
    \caption{xDSM diagram}
    \label{fig:n2}
\end{wrapfigure}

\lipsum[1]

As proposed in \cite{mccabe_system_2023}, CEM runs are performed ahead of time for a wide set of inputs designed to reflect the range of possible WEC designs and world scenarios, and the results are fit to a surrogate model that is used in the design optimization to reduce computation time.

\begin{wrapfigure}[18]{r}{0.52\textwidth}
    \centering
    \includegraphics[width=.5\textwidth]{figures/CEM-flowchart-idetc.png}
    \caption{Surrogate model diagram}
    \label{fig:surrogate}
\end{wrapfigure}

\subsubsection{MDOcean}

    -  cost
    -  power
    -  quick mention hydro, structures, amplitude limits, force saturation
\lipsum[1]

\subsubsection{Capacity Expansion Model (CEM)}

    -  GenX: equation~\eqref{eq:CEM-objective} for the LP it solves, what assumptions it makes (equilibrium)

\begin{equation}\label{eq:CEM-objective}
    \min_{x_{grid}} C_{grid} = \sum_{gen} \sum_{t} p_{gen,t} x_{grid,gen,t}
\end{equation}

    %-  time domain reduction and what timescales of power variation and demand alignment are captured vs relevant (seasonality, storms, within wave, within sea state between waves)
    -  scenarios / State of the world sweeps
    -  choice of spatial and temporal resolution and scope
    %-  Emissions in CEM
    %-  how we're confident enough that the results aren't due to random chance alignment between that day's demand and wave data and are real / significant
    -  GenX requires as input detailed data on the costs, hourly profiles, and existing infrastructure of all generation, transmission, storage, and load on the grid. This is managed by the PowerGenome package \cite{powergenome}, which consolidates data from multiple sources such as the U.S. Energy Information Administration (EIA) and National Renewable Energy Laboratory (NREL).
    \figureautorefname~\ref{fig:CEM-data-flow} depicts the data flow. Coefficients $\beta$ are found for each input combination of wave cost, grid scenario, location, damping ratio, and natural frequency. Orange outlines indicate code implemented in this work rather than an existing package. PowerGenome does not provide data on wave energy converters, so we use the MHKit interface to hindcast power densities \cite{mhkit} along with design-dependent capture width information to generate the hourly power profiles of the WECs. 

\subsubsection{Surrogate model to link CEM with design}
Linking the design and grid optimization requires first identifying a small number of key design characteristics to collapse the large WEC design space (reduced order model), and then building a surrogate model to predict the CEM outputs from these design characteristics.

For grid scenarios run in the CEM, the surrogate model can be replaced with a lookup table, but a true surrogate model permits limited extrapolation to different grid scenarios and understanding of driving factors.

12 design variables > 210 capture width vs Hs and T > 8760 power profile variables

Why am I not fitting CW(Hs,T)/CW_{max} = beta_0 + beta_1 Hs + beta_2 T + beta_3 Hs^2 + beta_4 T^2 + beta_5 Hs*T? That's more direct than even the Bode. It captures radiation efficiency (which itself includes amplitude limits, hydro shape, damping vs impedance control), drag efficiency, force saturation, electrical efficiency, and power limits.

In addition to the directly used CEM outputs of grid cost and emissions, we use additional CEM outputs (installed capacity of WECs and other generators) and compute indirect CEM 
, which is used as part of the surrogate model to predict the primary outputs from design inputs. 

For each grid scenario, we identify the maximum threshold cost representing the maximum cost of any generator (with the most favorable generation profile) that would be installed, and 



    -  idea of doing a bode fit and why it captures the right info
    -  bode fit figure is only valid for frequencies at and below the natural frequency and neglects change in $\zeta$ and $\omega_n$ with $H_s$%(equation~\eqref{eq:bode-fit}), 

%\begin{equation}\label{eq:bode-fit}
%\begin{aligned}
%    \zeta,\omega_n &= \mathrm{argmin}\left( \epsilon_{mag}^2 + \epsilon_{phase}^2 \right) \\
%    \epsilon_{mag} &= xxx \\
%    \epsilon_{phase} &= xxx
%\end{aligned}
%\end{equation}
    
    %- assumes that the WEC can generate less than its max power without violating any amplitude/force constraints        
    %-  how CEM system cost gets turned into NVOE metric (since CEM enforces NVOE=0)    

\begin{figure}[b]
\noindent
\begin{minipage}[t]{0.32\textwidth}
    \centering
    \includegraphics[width=\linewidth]{figures/PowerGenomeDataFlow.pdf}
    \captionof{figure}{Data flow for the CEM.}
    \label{fig:CEM-data-flow}
\end{minipage}
\hfill
\begin{minipage}[t]{0.32\textwidth}
    \centering
    \includegraphics[width=\linewidth]{figures/bode_second_order.pdf}
    \captionof{figure}{Bode fit}
    \label{fig:bode}
\end{minipage}
\hfill
\begin{minipage}[t]{0.32\textwidth}
    \centering
    \includegraphics[width=\linewidth]{example-image-a}
    \captionof{figure}{CEM output raw data and fit}
    \label{fig:three}
\end{minipage}
\end{figure}

\subsubsection{Life Cycle Analysis (LCA)}
    -  LCA items considered: steel, fiberglass,
    -  LCA items modeled but not changed in this optim: distance from shore
    -  LCA items not considered: fish, hydraulic fluid, detail design
    -  LCA table with weights
    -  If you express net eco value as a weighted sum of steel volume and surface area, how similar are the coeffs to the coeffs for cost?

\begin{table}[H]
    \begin{center}
    \begin{tabular}{ lll } 
     \hline
     Material/component & Eco-cost & Unit \\ 
     \hline
     Steel & xx & \$/kg \\ 
     Fiberglass & xx & \$/m\textsuperscript{2} \\ 
     Distance from shore & xx & \$/km \\ 
    \end{tabular}
    \caption{Eco-cost coefficients}
    \label{tab:lca-weights}
    \end{center}
\end{table}


\section{Results}
    -  CEM sweep (no optim) raw vs fitted
    -  single obj optim (min NVOE compared to min LCOE)
    -  multi obj optim (pareto NVOE vs net eco value per power)
    -  runtime info
    -  when WECs are viable, what are they replacing (batteries or wind or solar etc)? Find by looking at how does the optimal LP output change with
    -  can the CEM results be predicted by correlation coefficient of WEC power with demand or with energy storage or with lost load of solar, the capacity factor, sum of hourly profit, other proxies?

\begin{figure}[b]
\noindent
\begin{minipage}[t]{0.32\textwidth}
    \centering
    \includegraphics[width=\linewidth]{example-image-a}
    \captionof{figure}{CEM results: beta and confidence bounds for various grid scenarios}
    \label{fig:cem-results}
\end{minipage}
\hfill
\begin{minipage}[t]{0.32\textwidth}
    \centering
    \includegraphics[width=\linewidth]{example-image-a}
    \captionof{figure}{Comparison of single-objective optima}
    \label{fig:single-obj-compare}
\end{minipage}
\hfill
\begin{minipage}[t]{0.32\textwidth}
    \centering
    \includegraphics[width=\linewidth]{example-image-a}
    \captionof{figure}{Pareto front}
    \label{fig:pareto}
\end{minipage}
\end{figure}

\lipsum[1-3]

\section{Conclusions}
\lipsum[1]

- do we see WECs being viable NVOE-wise despite a high LCOE due to their temporal

- does optimizing for NVOE vs LCOE make a big difference in the optimal design?

\section*{Acknowledgments and Data Availability}
%%%%%%%%%%%%%%%%%%%%%%%%%%%%%%%%%%%%%%%%%%%%%%%%%%%%%%%%%%%%%%%%%%%%%%%%%%%%%%%%%%%%%%%%%%%%%%%%%%%%%%%%
The authors thank Olivia Vitale and Collin Treacy for feedback on a draft manuscript. The code for all simulation, optimization, analysis, and visualization to fully reproduce this work is available open-source: design optimization at \url{github.com/symbiotic-engineering/MDOcean} and capacity expansion model at \url{github.com/symbiotic-engineering/wec-decider}.

\clearpage
\section{Main Text}
\textbf{\textcolor[HTML]{70AD47}{Papers should be 3-5 pages in length, not inclusive of references. }} \hfill

\vspace{1\baselineskip}

%%%%%%%%%%%%%%%%%%%%%%%%%%%%%%%%%%%%%%%%%%%%%%%%%%%%%%%%%%%%%%%%%%%%%%%%%%%%%%%%%%%%%%%%%%%%%%%%%%%%%%%%
% \section*{Nomenclature}
%%%%%%%%%%%%%%%%%%%%%%%%%%%%%%%%%%%%%%%%%%%%%%%%%%%%%%%%%%%%%%%%%%%%%%%%%%%%%%%%%%%%%%%%%%%%%%%%%%%%%%%%
\begin{framed}
\textbf{Nomenclature} \\
\normalsize    
    A\ \ \ \ radius of \\    
    B \ \ \ \ position of \\    
    C\ \ \ \ further nomenclature continues down the page inside the text box \\
\end{framed}


%------------------------------------------------------------------------------------------------------%
\subsection{Tables}
%------------------------------------------------------------------------------------------------------%
Headings should be placed above tables, left justified. Only horizontal lines should be used within a table, to distinguish the column headings from the body of the table, and immediately above and below the table.

%------------------------------------------------------------------------------------------------------%
\subsection{Section headings}
%------------------------------------------------------------------------------------------------------%
Sub-section headings should be bold, in capital and lower-case italic letters, numbered 1.1, 1.2, etc., and left justified, with second and subsequent lines indented. 

%------------------------------------------------------------------------------------------------------%
\subsection{Footnotes}
%------------------------------------------------------------------------------------------------------%
Footnotes should be avoided if possible. Necessary footnotes should be denoted in the text by consecutive superscript letters\footnote{1 Footnote text.} \cite{mccabe_system_2023,mccabe_mdocean_2024}. The footnotes should be typed single spaced, and in smaller type size (8 pt), at the foot of the page in which they are mentioned, and separated from the main text by a one line space extending at the foot of the column.

%------------------------------------------------------------------------------------------------------%
\section{Illustrations}
%------------------------------------------------------------------------------------------------------%
Figures should be placed at the top or bottom of a page wherever possible, as close as possible to the first reference to them in the paper. The figure number and caption should be typed below the illustration in 8 pt and centered with the image. If two images fit next to each other, these may be placed next to each other to save space.


%%%%%%%%%%%%%%%%%%%%%%%%%%%%%%%%%%%%%%%%%%%%%%%%%%%%%%%%%%%%%%%%%%%%%%%%%%%%%%%%%%%%%%%%%%%%%%%%%%%%%%%%

%%%%%%%%%%%%%%%%%%%%%%%%%%%%%%%%%%%%%%%%%%%%%%%%%%%%%%%%%%%%%%%%%%%%%%%%%%%%%%%%%%%%%%%%%%%%%%%%%%%%%%%%


\vspace{1\baselineskip}

%%%%%%%%%%%%%%%%%%%%%%%%%%%%%%%%%%%%%%%%%%%%%%%%%%%%%%%%%%%%%%%%%%%%%%%%%%%%%%%%%%%%%%%%%%%%%%%%%%%%%%%%
\section*{Appendix A. CEM Surrogate Model Equation}
%%%%%%%%%%%%%%%%%%%%%%%%%%%%%%%%%%%%%%%%%%%%%%%%%%%%%%%%%%%%%%%%%%%%%%%%%%%%%%%%%%%%%%%%%%%%%%%%%%%%%%%%
$C$ refers to costs (\$M) (output from CEM), $S$ refers to specific capacity costs (\$M/kW) (input to CEM), $x_{grid}$ refers to installed capacities (decision variable in CEM), and $E$ refers to specific energy costs (\$/kWh). The amount of built wave capacity $x_{grid,WEC}$ is not to be confused with $x_{design}$, the WEC design variables in the outer (MDOcean) optimization that are parameters (held constant) in the CEM.

\begin{equation}
\begin{aligned}
    C_{grid}(p_{grid},x_{design}) &= C_{grid,0}(p_{grid})- \beta_1(p_{grid})~ x_{grid,WEC}(p_{grid},x_{design}) \\
    \frac{x_{grid,WEC}(p_{grid},x_{design})}{x_{grid,WEC}(p_{grid},x_{design})+\sum_{gen\neq WEC}x_{grid,gen}(p_{grid},x_{design})} &= \min\left(\beta_2(p_{grid}) \times \max\left(\frac{S_{WEC,thresh}(p_{grid},x_{design})}{S_{WEC}(x_{design})}-1,0\right), 1\right) \\
    S_{WEC,thresh}(p_{grid},x_{design}) &= S_{max,thresh}(p_{grid}) - \beta_3(p_{grid}) \zeta(x_{design}) - \beta_4(p_{grid}) \frac{\omega_n}{\omega_p}(x_{design}) - \beta_5(p_{grid}) \min\left(\frac{P_{max}}{P_{pk}}(x_{design}),1\right) \\
    \sum_{gen\neq WEC}x_{grid,gen}(p_{grid},x_{design}) &= \sum_{gen}x_{grid,gen,0}(p_{grid})-\beta_6(p_{grid}) \zeta(x_{design}) - \beta_7(p_{grid}) \frac{\omega_n}{\omega_p}(x_{design}) - \beta_8(p_{grid}) \min\left(\frac{P_{max}}{P_{pk}}(x_{design}),1\right)
\end{aligned}
\end{equation}

%improvements to make to above eqn:
%- debatable whether coeffs should be kept dimensional, ie should $\beta_1$ get divided by total capacity and $C_{grid,0}(p_{grid})$ so it's relating percent cost to percent capacity?
%- $\beta_{3-8}$ terms could be made nonlinear as needed - ie $(\frac{\omega_n}{\omega_p})^2-1$ shows up in theory. I could also try to feed-forward incorporate the time domain power reconstruction, ie a peak to average ratio, if that is looking like a better fit than the freq domain stuff, but that just tells me that my hope of doing freq domain to connect to mdocean won't work out.
%- ideally some of these betas turn out to be independent of p grid, which means more of the p grid dependence can be found just from the 0 (no wec) case for that p, without requiring a whole design sweep. That is the reason all this modeling is potentially more useful than a lookup table.
%- Could add saturation to make sure $S_{WEC,thresh}>0$ and $C_{grid}>0$.
%- add hats or some other notation to differentiate predicted (fit) from actual

\vspace{1\baselineskip}
\textit{A.1 Example of a sub-heading within an appendix}
\vspace{1\baselineskip}

There is also the option to include a subheading within the Appendix if you wish.

%%%%%%%%%%%%%%%%%%%%%%%%%%%%%%%%%%%%%%%%%%%%%%%%%%%%%%%%%%%%%%%%%%%%%%%%%%%%%%%%%%%%%%%%%%%%%%%%%%%%%%%%% References
%%%%%%%%%%%%%%%%%%%%%%%%%%%%%%%%%%%%%%%%%%%%%%%%%%%%%%%%%%%%%%%%%%%%%%%%%%%%%%%%%%%%%%%%%%%%%%%%%%%%%%%%

\printbibliography

\end{document}