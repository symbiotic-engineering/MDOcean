\section{Discussion}\label{sec:discussion}

Beyond the analysis of individual results offered in section~\ref{sec:results}, this section discusses the study's contributions, implications, and limitations more broadly.
It compares results to existing studies for context and suggests avenues for future work to resolve model shortcomings and answer additional design questions.

\subsection{Result Comparison to Other Studies }
% (and comparison to commercial WECs)
\subsubsection{Force Saturation}
Previous studies \cite{mcgilton_optimal_2024,coe_maybe_2021,devin_high-dimensional_2024,gaebele_tpl_2025,mccabe_force-limited_2024} suggest capping the maximum force significantly below the unsaturated point based on findings that doing so sacrifices only minimal power, but none perform full economic analysis to confirm that the cost savings outweigh the power decrease.
The results presented in section~\ref{sec:results-single} verify the conclusion with an appropriate cost analysis, since the optimal design sizes its powertrain small enough to necessitate force saturation.
The follow-on analysis in \sectionautorefname~\ref{sec:sensitivity-model-assumptions} repeats the optimization without allowing force saturation and finds a 5\% LCOE penalty.
Thus, force saturation as a control technique is highly valuable, capable of tangible LCOE reductions when considering the tradeoff of power and cost.
Future studies could quantify the threshold of the lowest marginal price of PTO force (\$/N) for which this conclusion still holds.

% \hl{add similar analysis for power saturation}

\subsubsection{Global Deployability}
A survey indicates that 40\% of WEC developers design their WEC to be deployment-site agnostic, so design for global deployability is of interest in industry \cite{trueworthy_wave_2020}.
A previous study \cite{de_andres_adaptability_2015} compares the global deployability across different wave climates of a uniform WEC design versus a tunable WEC design.
The study finds that tuning the resonant period to match the dominant ocean period by changing geometry increases the capture width ratio but decreases the more important power to mass ratio (an inverse proxy for LCOE), concluding that tuning is unfavorable because it requires expensive larger diameters.
Importantly, however, the devices have equal diameter and height, use only damping control, and are tuned to maximize power without regard for mass or cost.
In MDOcean, on the other hand, tuning can not only be accomplished by changing diameter and height independently, but also with reactive control.
Moreover, devices are tuned not to maximize power but to minimize LCOE.
The single-objective results in \tableautorefname~\ref{tab:opt-dv-values} and \ref{tab:opt-output} show agreement with the overall conclusion of \cite{de_andres_adaptability_2015} that power maximization does not align with LCOE minimization, but in the present study the additional cost of maximum power comes in the form of PTO cost rather than structural cost as it does in \cite{de_andres_adaptability_2015}.
Meanwhile, MDOcean's location global sensitivity (\tableautorefname~\ref{tab:location}) shows a significantly larger optimal float size in Hawaii, the location with the lowest value of the most common wave period.
This again departs from the results of \cite{de_andres_adaptability_2015}, which finds that small (4~m diameter) WECs optimize their LCOE proxy regardless of the dominant period at the location.
This demonstrates that the use of a more thorough multidisciplinary model like MDOcean is essential to draw accurate design conclusions.

\subsubsection{Structural Efficiency}
A previous study compares the structural mass per unit hull volume for marine structures including WECs and ship hulls, finding a value of $278.76 $~kg/m\textsuperscript{3} for steel structures \cite{roberts_bringing_2021}, with a 95\% confidence interval of $\pm280$ metric tons.
This trendline would predict the nominal float structural mass to be 72-632 metric tons and the minimum LCOE structural mass to be 1814-2374 metric tons.
MDOcean calculates structural masses of 213 and 80 metric tons respectively.
The nominal design lies within the confidence interval, but the optimized design lies far below it.
Even though the hull volume of the optimal design exceeds the hull volume of any data points used to create the fit, the large departure from existing data warrants concern.
In fact, the only data points with structural mass comparable to the optimized 80 ton float have a hull volume that is an order of magnitude lower than the optimized float, indicating that MDOcean may vastly underestimate the required thickness.
This requires further investigation and may be due to underestimation of loads in the dynamics module or overestimation of the factor of safety in the structures module.

% compare to the Sandia RM3 optimization results

\subsubsection{Optimization Procedure}
\label{sec:discussion-optimization}
Recalling the discussion of prior WEC optimization studies in section~\ref{sec:lit}, all prior large-scale ($\geq8$ design variables) WEC optimizations utilize the genetic algorithm, a population-based heuristic algorithm.
Based on the generation count and population sizes stated in references \cite{khanal_multi-objective_2024,garcia-teruel_reliability-based_2021,cotten_multi-objective_2022,abdulkadir_control_2024}, 12,000, 2,200, 6,100, and 2,400 function evaluations are required for convergence respectively.
The the present study is the first to perform large-scale WEC optimization with a gradient-based algorithm, which was expected to require fewer function evaluations to converge.
Generating the pareto seeds in this study, however, entails performing 20 single-objective optimizations at around 100 iterations each, using finite difference on the 12 design variables.
This requires an estimated $20\cdot100\cdot(12+1)=\paretoFcnEvals$ function evaluations, which dwarfs the function evaluations required by the genetic algorithm in other studies.
This result is surprising and can be attributed to the use of finite difference.
Despite the greater function evaluations, MDOcean's semi-analytical model means that the optimization likely runs faster than the other BEM-based optimizations, although the relevant papers do not include any runtime information that could verify or deny this claim.
Because of the fast runtime, effort was not invested to decrease the number of iterations, although moderate reductions should be possible by tuning convergence criteria, utilizing a gradient sparsity pattern to reduce the number of finite difference evaluations needed, and tuning the ratio of the number of seeds to the pattern search set size.
A more substantial decrease is possible by using automatic differentiation or the adjoint method to obtain derivatives.
These methods would make the number of function evaluations independent of the number of design variables, but they require more implementation effort.
For example, the adjoint method would additionally require constraint aggregation because of the large number of constraints, and automatic differentiation would require rewriting all functions in terms of elementary functions where possible and manually supplying gradients for unsupported functions such as Bessel functions in the hydrodynamics module.
Finally, even with the increased function evaluations, the gradient-based algorithm retains its advantage of providing certainty that the solution has converged to a local optimum, and with a systematic multistart, there is reasonable confidence that the solution is a global optimum, which is not the case in a heuristic optimization.

\subsection{Model Realism}
Circling back to the modeling assumptions introduced in section~\ref{sec:model}, one can now use the optimization and sensitivity results to determine whether the assumptions are reasonable.
We assess model complexities that MDOcean adequately captures to confirm whether those effects were worth including, as well as model shortcomings that MDOcean cannot adequately capture to examine whether those effects must be added in future versions.
Previous studies such as \cite{garcia-teruel_reliability-based_2021} have called for this kind of systematic model fidelity assessment to enhance designers' confidence on the most relevant effects and their impact on computation time.

\subsubsection{Observed Impact of Modeled Effects}
A few of the areas of developed model complexity, such as force saturation, slamming amplitude limits, and 2-DOF dynamics, stand out as contributing significantly to the final results, as seen in the constraint activity and difference in results obtained if 1-DOF dynamics corresponding to a stationary spar are used.
Section~\ref{sec:sensitivity-model-assumptions} analyzed these effects in detail.
Earlier in \sectionautorefname~\ref{sec:sim-runtime}, \figureautorefname~\ref{fig:runtime-dynamics} reveals that MDOcean's 2-DOF dynamics take an order of magnitude longer to run than 1-DOF dynamics, indicating an area where code enhancements to make the 2-DOF runtime more comparable with the 1-DOF model may be worthwhile given the dynamic importance of including the second degree of freedom. 
Meanwhile, force saturation is seen to be especially relevant since it not only directly reduces the powertrain cost, but also indirectly reduces the structural cost by reducing the operational heave loading.
This allows for a reduction in spar thickness due to the activity of the column fatigue factor of safety constraint. 
The observed impact of these effects on the overall optimization justifies the model complexity in those areas, suggesting that other optimization frameworks should include these complexities for realistic results.

On the other hand, a few other modeling complexities appear perhaps unnecessary in hindsight because they have little effect on the results.
Section~\ref{sec:sensitivities-local} finds a low sensitivity to drag coefficient, and the re-optimization without drag in \sectionautorefname~\ref{sec:sensitivity-model-assumptions} confirms this result.%\hl{confirms this result}.
This suggests that modeling the drag nonlinearity is not essential, and future optimizations may wish to turn off drag to reduce the computation time associated with iterating the drag damping coefficient.
Likewise, the slamming model is derived to be valid not just in the simple case of long waves but also in intermediate wavelengths, where the maximum amplitude depends on additional values including the phase of the dynamic response.
\appendixautorefname~\ref{sec:appendix-slam} shows that the simple slamming model is reasonably accurate at sea states with active slamming constraint, suggesting that the intermediate wavelength model may be unnecessary.
However, before making this claim confidently, it would be necessary to check whether this is true at all points during the optimization, not merely at the optimal point, because the slamming model complexity is not currently implemented as a model setting and the suitability of the long-wave condition at the optimum is checked manually. %\hl{I should actually check this.}
Finally, even though it is observed that the drag and intermediate wavelength slamming models have little effect on the results individually, the nonlinearity of the problem permits the possibility that a coupling effect may be observed if both were to be turned off simultaneously, or that an effect may emerge if the formulation were to change in another way such as optimizing another objective or adding more constraints.
Therefore, users should exercise caution if choosing to simplify the model.

\subsubsection{Potential Impact of Unmodeled Effects}
No constraints corresponding to priority 1 or 2 requirements are active in the optimal solution, % \hl{(double check this - structures effective breadth could make this false)}, 
a hopeful sign that the true optimum is unlikely to lie in a region of model inaccuracy.
Likewise, the inactivity of the pitch stability constraint mitigates the impact of any error in the metacentric height due to the geometry module's assumption of an even mass distribution when calculating the center of gravity (see section~\ref{sec:geom}).
On the other hand, the active slamming constraints for operational sea states mean that some assumptions are worth revisiting, particularly the use of an optimization constraint rather than enforcing amplitude limits with the constrained optimal linear controller (see tables \ref{tab:constraint-approaches} and \ref{tab:nonlinearities}), and the assumption that the WEC must operate in all sea states with nonzero probability in the JPD without going into survival mode (see ``Dynamic Limits and Design Load Cases'' in section~\ref{sec:dynamics}).
Additionally, while MDOcean accounts for the effect of the power limit on PTO cost, it does not directly model short-term power variation as would be necessary for sizing energy storage or grid connection, both of which would increase the effect of the power limit on cost.

\paragraph{Neglect of Surge Force}
Section~\ref{sec:dynamics} establishes that although the dynamics model estimates the surge force, the structures model uses only the heave force to calculate the factors of safety.
The neglect of surge force means that geometries with significant lateral areas that would experience high surge force perform better than they should in the optimization.
This is not only because of the impact on the structures module, but also because the surge force drives mooring cost, an effect which is not modeled in this study.
The minimum LCOE design experiences a surge force of \surgeForceFloatAtMinLCOE~on the float and \surgeForceSparAtMinLCOE~on the spar, compared to the nominal design's \surgeForceFloatNominal~and \surgeForceSparNominal~respectively.
With mooring and foundation comprising 12\% of the capital cost (see \tableautorefname~\ref{tab:CBS}), this effect may be substantial.
Incorporating surge force into the structures module and adding a mooring cost module are recommended for future work.
If these effects are implemented and confirmed to be important, then it may also be desired to obtain a more accurate estimate of the surge force across frequencies by utilizing the surge hydrodynamic coefficients instead of the long-wavelength approximation in \equationautorefname~\ref{eq:surge-force}. 

\paragraph{Regular, Linear Waves}
As section~\ref{sec:dynamics} describes, MDOcean uses standard linear wave theory with equivalent regular waves, even for the storm condition.
However, storm waves are nonlinear, and it is generally recommended to utilize higher order potential flow, CFD simulations, or wave tank tests to obtain storm loads \cite{coe_survey_2018}.
Additionally, even when storm wave analysis uses linear or quasi-linear hydrodynamics, other design procedures typically utilize a time-domain ``design wave elevation" signal to capture transient peaks.
The regular wave analysis used in MDOcean cannot capture nonsinusoidal peaks.
Therefore, determination of design loads from a given extreme sea state represents a major area of uncertainty in this model and a challenge for future development.
Open-source tools for WEC extreme response, such as MHKiT\footnote{\url{https://mhkit-software.github.io/MHKiT/}} and the DLC Generator\footnote{\url{https://dlc.primre.org/DLCGenerator}}, apply extreme statistics to facilitate the creation of a design wave elevation but do not calculate loads from this elevation.
Finding a modeling methodology that captures hydrodynamic nonlinearities and transient peaks without optimization-prohibitive computational expense remains an open problem.
Second-order MEEM can potentially address the former, but nonlinearities require solving the radiation and diffraction problems separately, and corner discontinuities present new analytical challenges \cite{cong_novel_2020,mavrakos_second-order_2009}.
Another option is a slender-body approximation for second order loads that was recently implemented in the frequency-domain quasi-linear hydrodynamics solver RAFT \cite{carmo_slender-body_2025}.
Meanwhile, to address the transient dynamics of an irregular design wave elevation, it requires more investigation to determine whether the semi-analytical dynamics model used here can be adequately extended.
Alternatives include integrating second-order hydrodynamics into pseudo-spectral methods \cite{coe_initial_2020}, or an effort to speed up time-domain models where second-order hydrodynamics have recently been integrated.\footnote{\url{https://github.com/WEC-Sim/WEC-Sim/pull/1242}}

Besides storm loads, the regular wave assumption also affects the operational power and load calculations.
In the future, the interaction of irregular waves with dynamic constraints could be roughly approximated in MDOcean with minimal implementation effort by computing the probability density function for each signal and saturating any amplitude above the constraint threshold.
To more fully capture nonlinearities in irregular waves, the describing function mentioned earlier (which quantifies the fundamental amplitude of the response to a deterministic sinusoidal input) could be replaced with its probabilistic counterpart, stochastic linearization (which quantifies the expected value of the fundamental over the spectral input).
This technique has been explored in several WEC papers \cite{da_silva_statistical_2020,da_silva_stochastic_2023,kluger_synergistic_2017,folley_spectral-domain_2016,spanos_efficient_2016}, including one that performs multi-objective design optimization \cite{neshat_enhancing_2024}.

% other unmodeled effects to talk about later:
% MEEM damping plate, especially for spar added mass vs frequency
% MEEM non-flat bottoms
% DF for power limit
% Real PTO model
% structures model - adding local buckling, doing structural FEA validation, allowing for more prominent stiffeners using effective breadth

\subsection{Formulation Realism}
\subsubsection{Design Variable Suitability}
Section~\ref{sec:formulation} outlined this study's careful choice of design variables to facilitate satisfaction of high-priority requirements to maintain model validity and facilitate optimization convergence while balancing the computational challenges of a large design space.
The analysis in \sectionautorefname~\ref{sec:sensitivities-local} reveals that the following variables chosen as parameters may be worthy of including as design variables in future studies: $D_d$, $T_{f,1}$, $T_s$, and possibly $D_{f,in}$.
In addition, the addition of $N_{WEC}$ as a design variable carries little importance in an LCOE minimization, where larger $N_{WEC}$ should always be preferred due to economies of scale, but can offer especially useful insight in a modified multi-objective analysis where the first objective is changed from per-WEC average power to array average power.
This is relevant for small-scale Powering the Blue Economy applications with a fixed power requirement, where it is unclear whether it preferable to meet that requirement using multiple smaller WECs or one larger WEC.

Additionally, a sensitivity analysis of design variables perturbed as if they are parameters would help to analyze the extent of coupling between disciplines and potentially inform future formulation simplifications.
For example, the expected high sensitivity of the optimal structural design variables to the powertrain design variables $\left(\frac{dx^*_{struct}}{dx_{pto}}\right)$ aligns with the choice made in this study to conduct fully coupled MDO, as outlined in section~\ref{sec:model-overview}, but a low value would support the potential disciplinary decoupling.
This type of analysis may be necessary if a proposed future optimization with significantly more design variables is too computationally intensive to perform fully coupled optimization and requires splitting up into a set of sequential optimizations, such as bulk geometry power maximization followed by powertrain sizing followed by structural sizing).
Performing this sequential optimization for the current formulation and assessing the degree of suboptimality could quantify the importance of a coupled optimization.
The authors predict, however, that the significant computational speedup of the semi-analytical models introduced in this paper and the extensions discussed herein will be sufficient to allow full coupling even for substantially larger scale WEC optimizations, including the ambitious target of a study comparing multiple hydrodynamic architectures with realistic powertrain models.

\subsubsection{Parameter Suitability}
The local parameter sensitivity analysis of \sectionautorefname~\ref{sec:sensitivities-local} already yields significant insight on the impact of the choice of parameters in this problem.
Interpreting those results more broadly, parameters with a high design sensitivity $\left(\frac{dx^*}{dp}\right)$, such as dimensional aspect ratios, wave conditions, and material properties, are especially important for engineers to determine early in the design cycle because they affect the optimal design. 
Where relevant, designers should consider holding the nondimensional versions of these parameters constant across prototype scales, extending the common practice of Froude scaling for wave conditions.
On the other hand, parameters with a high objective sensitivity $\left(\frac{dJ^*}{dp}\right)$, such as operational wave conditions, PTO efficiency, and financing rates in the LCOE minimization, are most important for funders to determine early in the project implementation cycle and for research funding agencies to incentivize innovation through R\&D proposals because they affect the WEC's economic viability.
Notably, other parameters such as PTO price, storm wave conditions, and dimensional variables are more relevant for design cost minimization, highlighting that incentives and innovation strategies for Powering the Blue Economy applications may need to be distinct from those for utility-scale energy.
Meanwhile, the larger number of parameters with high sensitivity in the cost minimization implies that there are more diverse avenues for impactful innovation in PBE applications than in utility-scale energy.

Additional sensitivity and post-optimality analysis work that could produce useful insights includes a sensitivity analysis of hyperparameters, following the same procedure as section~\ref{sec:sensitivities-local}, to ensure numerical convergence and adequate tuning of optimization settings.
A global parameter sensitivity analysis is also suggested, taking into account both nonlinearites in the effect of one parameter across operating points and interaction effects between multiple uncertain parameters.

\subsubsection{Constraint and Requirement Suitability}
By including \numLinConstraints~linear and \numNonlinConstraints~nonlinear constraints, this optimization captures more realistic design, model, and operational requirements than any other WEC optimization study known to the authors.
The active constraints highlighted in \figureautorefname~\ref{fig:constraint-activity} fall broadly into the class of survivability, including both structural (design variable lower bounds and factors of safety in each of the float, column, and damping plate) and amplitude-based (float rising above spar, linear theory, and float slamming). 
Constraints related to ballast and pitch stability are inactive, confirming that these design considerations are not primary avenues for innovation.
While survivability is known to be a critical design requirement for WECs, it is rarely incorporated into optimization studies, and the results of this study demonstrate and quantify the effect of survivability considerations. 
For example, the Lagrange multipliers of \figureautorefname~\ref{fig:lagrange-multiplier} suggest that improvements to the structural design of the damping plate are worth around 5 times more than equivalent structural improvements to the column and 10 times more than those to the float.

Further innovation in WEC survivability is warranted, both in the incremental and transformative sense.
Incremental improvements include developing more structurally and hydrodynamically performant detail designs that obey these constraints while reducing material use and sustaining energy production (for example, performing FEA or adding additional stiffeners to justify the removal of more material, replacing welded connections with pins to reduce moment transfer, or modifying the float-spar attachment geometry to allow for more travel with the same bulk body dimensions).
Transformative improvements include creative design concepts that avoid the need for these constraints altogether or substantially modify their formulation.
For example, storm survival strategies that involve submerging the WEC, changing the shape using control surfaces, or changing the PTO control strategy to reduce force and amplitude have all been suggested, and the viability of such design changes can be readily estimated by comparing the additional costs (for example, of additional actuator and control hardware and software) against the performance gains again using the Lagrange multipliers of \figureautorefname~\ref{fig:lagrange-multiplier}.

That said, there still exist many considerations that the formulation fails to capture, requiring caution before asserting that the industry should pursue the optimal design found here.
For one, design variables are assumed continuous, whereas in reality the sizing of structural thicknesses and, to a lesser extent, generator specifications, are constrained based on the discrete sizes available commercially.
Additional requirement omissions that could reasonably be considered in future work include mooring loads, structural resonant mode frequencies, and overturning moment in the storm design load case.
Requirements like installation and manufacturing feasibility, environmental and regulatory acceptability, reliability, and social benefit are more difficult to incorporate in the present modeling framework but nonetheless important.
For example, maximum dimensions may be limited by roadways, as is already the case for wind turbines, constraining maximum dimensions to 14-15 feet (4.3-4.6 m) \cite{cotrell_analysis_2014}.
Assessing the relationship between maximum component size (as transported) and maximum WEC dimensions (as installed) requires additional manufacturability and on-site assembly analysis beyond the scope of the present study.
To guide future work on requirements and constraints in WEC design optimization, the reader is referred to \cite{bull_systems_2017,babarit_stakeholder_2017} for a comprehensive breakdown and mapping of WEC requirements, and to \cite{trueworthy_wave_2020} for a study of how developers prioritize and address WEC design requirements in practice.

\subsubsection{Objective Suitability}
Due to the high uncertainty inherent in LCOE, this study performs multi-objective optimization on cost and power production and includes post-optimality analysis such as local and limited global sensitivity analysis.
While this overcomes many concerns in optimizing LCOE, there are still limitations.
For example, prices for steel, powertrain force, and powertrain power must still be assumed, which could be overcome by increasing the number of objectives to consider each cost separately.
However, moving beyond two objectives limits the intuition and visualization power of results, and the authors believe that the two objectives chosen here represent a good balance.
Alternatively, a global parameter sensitivity analysis of the price parameters could address this uncertainty. Pairing this with uncertainty quantification and robustness analysis would complete the picture.

Although the cost formulation used here is uncertain, it still represents a significant improvement over cost proxies typically used in WEC optimization such as wetted surface area and hull volume.
In future work, running the MDOcean optimization using those simplified metrics instead of the structural and powertrain cost model used here would be useful for comparison and to assess the suitability of various proxy metrics. 

% 	\paragraph{Not including powertrain impedance/generator model/real control co-design

% 	\paragraph{Grid scale does not include PBE}

%\paragraph{Does this design have high TPL? What are the risks and uncertainties?}
%use \url{https://tpl.nrel.gov/assessment} and \url{https://www.osti.gov/biblio/1825239} for RM3 TPL.

\begin{table}
    \centering
    \begin{tabular}{c|>{\centering\arraybackslash}p{0.4\linewidth}>{\centering\arraybackslash}p{0.4\linewidth}}
         &  Existing optimization formulation& New optimization formulation\\ \hline
         Existing model features&  Examples of changes that would make it even faster:
1) use gradient sparsity pattern to get rid of FD evaluations, or get rid of FD altogether by using matlab automatic differentiation 
2) make multi body code simpler more like single body& 
         
         Example design studies that could be done with the model more-or-less as it is now:

	1) more rigorously determine the effect of subsystem coupling - compare to sequential optimization of hydro then PTO sizing then structural - would let you know the benefit of coupled optimization for each subsystem, if it is large enough to justify the extra computational cost

	2) survivability - compare PTO-free vs PTO-braked in storm, to answer question of which type of survival mode is preferable; excitation loads using MEEM when float is sunk (for now could approximate as float with larger draft, once weird-region MEEM exists use that); allow decision on which sea states are operational vs survival could answer the question of whether it makes economic sense to go into survival mode more often. 

	3) more detailed PTO study: include a drivetrain with linear dynamics reflecting spring, flywheel, gear ratio, etc. -  If you assume the flywheel inertia and spring stiffness aren't controllable per sea state, would be useful as-is for finding the best inertia/stiffness across sea states to pair with the controller that can adjust per sea state.
But if you assume adjustable drivetrain, as has been achieved with magnetic spring (cite), this would either need to introduce constraints or a cost term related to these drivetrain parameters, since otherwise the solution should go to zero damping and stiffness equal to what the unconstrained optimal control stiffness is.
Could answer the interesting question of how much an adjustable PTO is worth over a non-adjustable one.
Could also provide a preliminary answer to the question of optimum gear ratio (low GR means high generator torque so more force saturation for a given generator, but less friction.
Vs high gear ratio has less force saturation but higher friction), although a generator model would be needed to more conclusively answer the question (discussed below).

    4) Add $N_{WEC}$ as a design variable and see when it's preferable to do many smaller WECs vs one bigger WEC 
    
    5) How do my results compare to optimizing for easier cost proxies (ie structural cost as surface area vs actual force-informed thicknesses)\\
         
         New model features&  Examples of model enhancements that would not really unlock new types of design studies on their own, but would make this and any other design study more accurate and trustworthy/realistic:

	1) getting force with different nonlinear method: would make storm force more accurate, important since that is probably the least-theortically-defensible assumption of MDOcean

	2) more structures features: could use fatigue lifetime and in econ have lifetime = min(p.lifetime, fatigue lifetime) to be less conservative in structural sizing (currently uses endurance limit, which implies infinite fatigue lifetime)

    3) drag integral

    4) static friction describing function - ability to model static friction in drivetrain, and hydraulic check valves

    5) use extreme statistics to assess amplitude constraints, instead of max nonzero jpd& Example model additions and the types of design studies they would unlock:

	1) Multibody: would let me actually optimize the spar dimensions, which is important since I'm up against that constraint of min damping plate diameter

	2) Irregular waves: would let you have more realistic fatigue (ie Miner's rule), and assess power variability (ie with energy storage), which would let you see if variability reductions achieved via design are more preferable than variability reductions achieved via more batteries.
Would also give more complete results for PTO impedance matching bandwidth, ie are some PTO types or WEC shapes better Z(omega) shape matches with each other or with the sea state.

	3) model different WEC archetypes: would let you compare different WEC archetypes in a consistent way in the early concept design phase and start to address the design convergence problem.
Would require not only MEEM for the different WEC types, but also semi-analytical structural models for each one, or the decision to integrate a lightweight FEA with plate/shell element capabilities (ie pynite).

	4) make lifetime a design variable, and adjust size of storm-wave accordingly. could address the question of what opex cost threshold would make it useful to intentionally design 'single-use' WECs, which might be considered in PBE applications.
Would require integration with WDRT or similar to get the environmental contour for a given lifetime.

	5) multi region MEEM: could not only make hydro coeffs more accurate representing the truncated cone shape, but would allow qualitatively different hydro coeff vs frequency shapes, which would unlock different 

	6) generator physics and model - building off the PTO CCD study that is easily realizable with the current model, a generator model would unlock control co-design with the generator itself, for example whether a high-torque (expensive) generator is worth it compared to a low torque generator.
The torque limit would essentially impose a constraint on the $F_{max}$ design var, the core losses introduce nonlinear damping, and impose a relation between generator torque limit and max/min generator inertia.
A simplified generator model is explored in the RM3 CCD study \cite{anderson_re-imagining_2024}, or a full generator model as in the offshore wind MDO study \cite{barter_beyond_2023}.

7) structures constraint that accounts for other types of materials (ie brittle fracture (Mohr's circle) for concrete, however inflatables are modeled, composites, inflatable polyurethane coated nylon fabric, which \cite{roberts_bringing_2021} found to be more optimal than steel/reinforced concrete/fiberglass/rubber).
My relations between force and stress should still be good, but the relation between stress and FOS (the material limit state) is what needs to change.\\
    \end{tabular}
    \caption{Future model and optimization improvements}
    \label{tab:future-studies}
\end{table}



