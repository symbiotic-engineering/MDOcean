\clearpage
\section{Validation and Benchmarking}
\label{sec:validation-benchmarking}
\subsection{Dynamic Validation using WEC-Sim}\label{sec:appendix-wecsim}
The popular time-domain hydrodynamic simulation software WEC-Sim \cite{ruehl_wec-simwec-sim_2024} is used to validate the dynamics module.
The WEC-Sim RM3 example is run with regular waves and with the device constrained to oscillate only in heave.
Notably, the RM3 geometry provided in WEC-Sim differs slightly from the published dimensions in the RM3 report, so for validation the dimensions input to MDOcean are adjusted to match WEC-Sim.

%Both fixed-spar and floating spar configurations are tested to understand the implications of MDOcean's fixed-spar assumption.
WEC-Sim runs utilize hydrodynamic coefficients obtained with the WAMIT BEM for dynamics, and control coefficients calculated with MDOcean for consistency.
MDOcean is run with MEEM as usual, and separately also run with the WAMIT coefficients % and with a modified version of the WAMIT coefficients with zero excitation phase.
to distinguish differences caused by disparate hydrodynamic coefficients from those caused by the underlying dynamics. 

The error in average power compared to the WEC-Sim power is less than 0.1\% in the best case and 38.2\% in the worst case, with an error breakdown for all simulation scenarios and sea states shown in Figure~\ref{fig:error-histogram}. 

\begin{figure}
    \centering
    \includegraphics[width=1\linewidth]{\matlabFilepath{10}}
    \caption{Error breakdown based on WEC-Sim Validation Runs}
    \label{fig:error-histogram}
\end{figure}
% Results reveal that the assumption of zero excitation phase ($\gamma=|\gamma|$) has no effect for single body dynamics in the absence of drag, extremely minor effect (0.34\% annual energy) for fixed-spar dynamics in the presence of drag, and major effects for floating-spar dynamics. 

Results reveal that the drag describing function and MEEM hydrodynamic coefficients have a minor effect assuming a 1-DOF system (9.7\% and 2.7\% error on the average power and maximum amplitude respectively) but a major effect on the 2-DOF system (38.2\% and 28.6\% respectively).
The extremely low errors in the 2-DOF system enforcing the same hydrodynamic coefficients as WEC-Sim and with zero drag (0.2\% and 1.9\% in power and amplitude respectively) indicate that this is not an error in the 2-DOF dynamic model itself, but in the way that a 2-DOF model amplifies errors in drag and hydrodynamic coefficients due to the importance of the phase of motion between each DOF. 
These errors are deemeed acceptable for the purposes of this study, since the goal is to demonstrate the value of simultaneously optimizing multiple disciplines and the ability to quickly evaluate a large number of design options.
More details on the WEC-Sim validation are described in \appendixname~\ref{sec:appendix-dynamic-validation}.

\subsection{Static Validation}

Figure \ref{fig:meem-hydro-coeffs} shows the hydrodynamic coefficients as a function of wave frequency $\omega$ for the nominal RM3 geometry. 
\begin{figure}
\centering
\includegraphics[width=\linewidth]{\matlabFilepath{6}}
\caption{Hydrodynamic coefficients vs. frequency}\label{fig:meem-hydro-coeffs}
\end{figure}
The MEEM results for the simplified cylindrical geometry are a close match to WAMIT BEM results for the exact RM3 geometry, except at frequencies below 0.3 rad/s.
This is expected because the BEM solution assumes infinite depth, which requires a finite added mass at low frequency, whereas MEEM uses finite depth, where the added mass must grow logarithmically toward infinity at zero frequency \cite{mciver_added_1991}.
For the site under consideration, the lowest-frequency sea state containing any energy is 0.4~rad/s, so the discrepancy is not a concern.


The overall model was validated by comparing simulated structural forces, power, mass, and cost results to the nominal values in \cite{RM3}, as shown in Table \ref{tab:validation}.
The mass, power, and cost track well, but the structural force has a significant discrepancy.
This is because load cases in \cite{RM3} are derived experimentally from wave tank tests rather than a model.
For this study, a scale factor on force was used to account for the discrepancy.
Improving the force model to align with the wave tank data is an area of future work.

%\hl{Describe the radiation and amplitude limit and show they aren't violated} \cite{zou_practical_2023}.

%\hl{Describe how there are two different RM3 designs}

% \begin{table}[]
% \centering
% \begin{tabular}{l|l|l|l|l}
%                                                      & \multicolumn{2}{l|}{WEC-Sim RM3 Design}                                      & \multicolumn{2}{l}{DOE Report RM3 Design \cite{RM3}}                               \\
%                                                      Variable& MDOcean                    & Actual                     & MDOcean                    & Actual                            \\ \hline
% Float mass                     &                            &                            & 202.4 Mg                   & 208 Mg                            \\
% Vertical column mass           &                            &                            & \hl{fail}                  &                                   \\
% Reaction plate mass            &                            &                            & 230.9 Mg                   & 245 Mg                            \\
% Total mass                     &                            &                            & 614.3 Mg                   & 680 Mg                            \\
% Float volume                   & 726.8 m\textsuperscript{3} & 725.8 m\textsuperscript{3} & 701.9m\textsuperscript{3}  & 701.97m\textsuperscript{3}        \\
% Spar volume                    & 887.8 m\textsuperscript{3} & 886.7 m\textsuperscript{3} & 1,007 m\textsuperscript{3} & 1,008 m\textsuperscript{3}        \\
% CAPEX                          &                            &                            & \hl{fail}                  &                                   \\
% OPEX                           &                            &                            & [1.0 e6, \hl{...}]         & [1.2, 3.3, 6.6, 9.4] $\cdot 10^6$ \\
% LCOE                           & \hl{fail}                  &                            & \hl{fail}                  &                                   \\
% Average power                  &                            &                            &                            &                                   \\
% Heave force                    &                            &                            & 8.497 MN                   & 8.5 MN                            \\
% Spar factor of safety          &                            &                            & 11.894                     & 11.1\textsuperscript{*}           \\
% Power coeff. of variation& \hl{fail}                  &                            & 76.7\%                     & 71.1\%                            \\
% Float center of buoyancy       & 1.293 m                    & 1.293 m                    &                            &                                   \\
% Float center of gravity        & 0.283 m                    & 0.283 m                    &                            &                                  
% \end{tabular}
% \caption{Validation}
% \label{tab:validation}
% \end{table}

\begin{table}[]
\centering
\input{ \tableFilepath{12} }
\caption{Validation}
\label{tab:validation}
\end{table}

%\hl{Explain sources of error and rough uncertainty and the implications of what we can trust}

The scaling behavior of economic outputs against the number of WECs is validated in \figurename~\ref{fig:econ-nwec-validate}.
\begin{figure}
    \centering
    \includegraphics[width=\linewidth]{\matlabFilepath{12}}
    \caption{Validation for cost scaling with number of WECs}
    \label{fig:econ-nwec-validate}
\end{figure}

\subsection{Runtime Benchmarking}
\label{sec:sim-runtime}
Benchmarking the runtime of the MDOcean simulation is important to verify it achieves the desired speed to facilitate rapid design optimization.
An initial speed requirement order of magnitude of 100~ms for all modules was set to enable a 100-iteration finite difference optimization with 12 design variables to complete in around two minutes.
Ultimately, each simulation run takes around \simRuntime~ms, solidly meeting the goal.
The timings in this section are performed on an Ubuntu 20.04 LTS server with a 14-core Intel Core i9-10940X CPU (3.3 GHz base clock) and 256 GB of DDR4 RAM at 3200 MHz, running MATLAB R2024b.

Figure~\ref{fig:runtime-modules} visualizes the breakdown of runtime between modules.
\hl{Note: these figures are created using profiler, which dramatically overestimates all runtimes and can only be relied on for relative timing. This will be solved during re-scrutineering by using the timeit function instead.}
The MEEM hydrodynamics module takes the majority (\pctRuntimeMEEM) of the time and is broken down in \figureautorefname~\ref{fig:runtime-hydro}.
The biggest portion is dedicated to evaluating Bessel functions which occur in the semi-analytical solution, another large portion is spent unpacking variables from the cell data structure, and a smaller period solves the imaginary modes of the dispersion relation and solving the linear matrix equation.
The simulation is an order of magnitude faster than the Capytaine boundary element method solver for similar convergence levels.

\begin{figure}
\centering
\includegraphics[width=\linewidth]{\matlabFilepath{13}}
\caption{Bar chart showing simulation runtime breakdown between modules}\label{fig:runtime-modules}
\end{figure}

\begin{figure}
\centering
\includegraphics[width=\linewidth]{\matlabFilepath{13_1}}
\caption{Bar chart demonstrating the speed improvement of MDOcean's hydro module over baseline solver Capytaine}\label{fig:runtime-hydro}
\end{figure}

The dynamics and controls module takes the next longest (\pctRuntimeDynamics, enlarged in \figureautorefname~\ref{fig:runtime-dynamics}), with contributions from force saturation, spar analysis, drag linearization, and evaluation of the motion transfer function.
This represents a three order of magnitude improvement over the equivalent regular-wave WEC-Sim simulation run in parallel.
Simplifying the dynamics to a single degree of freedom (DOF) achieves another order of magnitude speedup, although the optimization and benchmarking results presented here utilize the 2-DOF model.

\begin{figure}
\centering
\includegraphics[width=\linewidth]{\matlabFilepath{13_2}}
\caption{Bar chart demonstrating the speed improvement of MDOcean's dynamics module over baseline solver WEC-Sim}\label{fig:runtime-dynamics}
\end{figure}

The structures, geometry, and economics modules are not computationally expensive and together compose the remaining \pctRuntimeOther~of the total runtime. 

% say that it's 210 sea states, that wecsim is parallelized across X cores, that MEEM is using N=M=K=10
%\hl{Describe the implication of how accurate my model is for being so fast - yay}

