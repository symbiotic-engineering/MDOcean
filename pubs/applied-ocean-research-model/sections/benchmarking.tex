\section{Runtime Benchmarking}
\label{sec:sim-runtime}
Benchmarking the runtime of the MDOcean simulation is important to verify it achieves the desired speed to facilitate rapid design optimization.
An initial speed requirement order of magnitude of 100~ms for all modules was set to enable a 100-iteration finite difference optimization with 12 design variables to complete in around two minutes.
Ultimately, each simulation run takes around \simRuntime~ms, solidly meeting the goal.
The timings in this section are performed on an Ubuntu 20.04 LTS server with a 14-core Intel Core i9-10940X CPU (3.3 GHz base clock) and 256 GB of DDR4 RAM at 3200 MHz, running MATLAB R2024b.

Figure~\ref{fig:runtime-modules} visualizes the breakdown of runtime between modules.
\hl{Note: these figures are created using profiler, which dramatically overestimates all runtimes and can only be relied on for relative timing. This will be solved during re-scrutineering by using the timeit function instead.}
The MEEM hydrodynamics module takes the majority (\pctRuntimeMEEM) of the time and is broken down in \figureautorefname~\ref{fig:runtime-hydro}.
The biggest portion is dedicated to evaluating Bessel functions which occur in the semi-analytical solution, another large portion is spent unpacking variables from the cell data structure, and a smaller period solves the imaginary modes of the dispersion relation and solving the linear matrix equation.
The simulation is an order of magnitude faster than the Capytaine boundary element method solver for similar convergence levels.

\begin{figure}
\centering
\includegraphics[width=\linewidth]{\matlabFilepath{13}}
\caption{Bar chart showing simulation runtime breakdown between modules}\label{fig:runtime-modules}
\end{figure}

\begin{figure}
\centering
\includegraphics[width=\linewidth]{\matlabFilepath{13_1}}
\caption{Bar chart demonstrating the speed improvement of MDOcean's hydro module over baseline solver Capytaine}\label{fig:runtime-hydro}
\end{figure}

The dynamics and controls module takes the next longest (\pctRuntimeDynamics, enlarged in \figureautorefname~\ref{fig:runtime-dynamics}), with contributions from force saturation, spar analysis, drag linearization, and evaluation of the motion transfer function.
This represents a three order of magnitude improvement over the equivalent regular-wave WEC-Sim simulation run in parallel.
Simplifying the dynamics to a single degree of freedom (DOF) achieves another order of magnitude speedup, although the optimization and benchmarking results presented here utilize the 2-DOF model.

\begin{figure}
\centering
\includegraphics[width=\linewidth]{\matlabFilepath{13_2}}
\caption{Bar chart demonstrating the speed improvement of MDOcean's dynamics module over baseline solver WEC-Sim}\label{fig:runtime-dynamics}
\end{figure}

The structures, geometry, and economics modules are not computationally expensive and together compose the remaining \pctRuntimeOther~of the total runtime. 

% say that it's 210 sea states, that wecsim is parallelized across X cores, that MEEM is using N=M=K=10
%\hl{Describe the implication of how accurate my model is for being so fast - yay}

