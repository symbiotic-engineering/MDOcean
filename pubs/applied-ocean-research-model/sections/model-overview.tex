\section{Modeling Methodology}
\label{sec:model}

This paper places considerable emphasis not only on the specific methods used but also on the process to select these methods, informed through a lens of MDO and systems thinking, with the intent to guide readers interested in conducting their own multidisciplinary WEC optimization.
In section~\ref{sec:model-overview}, the research objective and design scope are first used to define a broad problem formulation and module decomposition.
Next, section~\ref{sec:modules} describes the construction of the simulation model.
It develops the assumptions, analysis method, and implementation details of each module based on the appropriate balance of accuracy and speed.
Sections \ref{sec:validation} and \ref{sec:sim-runtime} discuss model validation and runtime benchmarking.
The simulation capabilities and limitations are then combined with the original broad design optimization intent to inform a detailed optimization problem formulation.
In this step, the exact objectives, design variables, constraints, and parameters are selected and outlined in section~\ref{sec:formulation}.
The resulting optimization problem is then characterized and its properties ultimately inform the solution method including algorithm selection, discussed in section~\ref{sec:optim-process}.
Finally, the process is tweaked iteratively until it yields satisfactory results, with noteworthy insights integrated into the relevant sections \ref{sec:model-overview} to \ref{sec:optim-process}.
Figure~\ref{fig:overview-methods} depicts this methodology visually.
\begin{figure}
    \centering
    \includegraphics[width=1\linewidth]{\matlabFilepath{2}}
    \caption{Methodology overview}
    \label{fig:overview-methods}
\end{figure}

\subsection{Overview}
\label{sec:model-overview}
\section{Modeling Methodology}
\label{sec:model}

This paper places considerable emphasis not only on the specific methods used but also on the process to select these methods, informed through a lens of MDO and systems thinking, with the intent to guide readers interested in conducting their own multidisciplinary WEC optimization.
In section~\ref{sec:model-overview}, the research objective and design scope are first used to define a broad problem formulation and module decomposition.
Next, section~\ref{sec:modules} describes the construction of the simulation model.
It develops the assumptions, analysis method, and implementation details of each module based on the appropriate balance of accuracy and speed.
Sections \ref{sec:validation} and \ref{sec:sim-runtime} discuss model validation and runtime benchmarking.
The simulation capabilities and limitations are then combined with the original broad design optimization intent to inform a detailed optimization problem formulation.
In this step, the exact objectives, design variables, constraints, and parameters are selected and outlined in section~\ref{sec:formulation}.
The resulting optimization problem is then characterized and its properties ultimately inform the solution method including algorithm selection, discussed in section~\ref{sec:optim-process}.
Finally, the process is tweaked iteratively until it yields satisfactory results, with noteworthy insights integrated into the relevant sections \ref{sec:model-overview} to \ref{sec:optim-process}.
Figure~\ref{fig:overview-methods} depicts this methodology visually.
\begin{figure}
    \centering
    \includegraphics[width=1\linewidth]{\matlabFilepath{2}}
    \caption{Methodology overview}
    \label{fig:overview-methods}
\end{figure}

\subsection{Overview}
\label{sec:model-overview}
\section{Modeling Methodology}
\label{sec:model}

This paper places considerable emphasis not only on the specific methods used but also on the process to select these methods, informed through a lens of MDO and systems thinking, with the intent to guide readers interested in conducting their own multidisciplinary WEC optimization.
In section~\ref{sec:model-overview}, the research objective and design scope are first used to define a broad problem formulation and module decomposition.
Next, section~\ref{sec:modules} describes the construction of the simulation model.
It develops the assumptions, analysis method, and implementation details of each module based on the appropriate balance of accuracy and speed.
Sections \ref{sec:validation} and \ref{sec:sim-runtime} discuss model validation and runtime benchmarking.
The simulation capabilities and limitations are then combined with the original broad design optimization intent to inform a detailed optimization problem formulation.
In this step, the exact objectives, design variables, constraints, and parameters are selected and outlined in section~\ref{sec:formulation}.
The resulting optimization problem is then characterized and its properties ultimately inform the solution method including algorithm selection, discussed in section~\ref{sec:optim-process}.
Finally, the process is tweaked iteratively until it yields satisfactory results, with noteworthy insights integrated into the relevant sections \ref{sec:model-overview} to \ref{sec:optim-process}.
Figure~\ref{fig:overview-methods} depicts this methodology visually.
\begin{figure}
    \centering
    \includegraphics[width=1\linewidth]{\matlabFilepath{2}}
    \caption{Methodology overview}
    \label{fig:overview-methods}
\end{figure}

\subsection{Overview}
\label{sec:model-overview}
\section{Modeling Methodology}
\label{sec:model}

This paper places considerable emphasis not only on the specific methods used but also on the process to select these methods, informed through a lens of MDO and systems thinking, with the intent to guide readers interested in conducting their own multidisciplinary WEC optimization.
In section~\ref{sec:model-overview}, the research objective and design scope are first used to define a broad problem formulation and module decomposition.
Next, section~\ref{sec:modules} describes the construction of the simulation model.
It develops the assumptions, analysis method, and implementation details of each module based on the appropriate balance of accuracy and speed.
Sections \ref{sec:validation} and \ref{sec:sim-runtime} discuss model validation and runtime benchmarking.
The simulation capabilities and limitations are then combined with the original broad design optimization intent to inform a detailed optimization problem formulation.
In this step, the exact objectives, design variables, constraints, and parameters are selected and outlined in section~\ref{sec:formulation}.
The resulting optimization problem is then characterized and its properties ultimately inform the solution method including algorithm selection, discussed in section~\ref{sec:optim-process}.
Finally, the process is tweaked iteratively until it yields satisfactory results, with noteworthy insights integrated into the relevant sections \ref{sec:model-overview} to \ref{sec:optim-process}.
Figure~\ref{fig:overview-methods} depicts this methodology visually.
\begin{figure}
    \centering
    \includegraphics[width=1\linewidth]{\matlabFilepath{2}}
    \caption{Methodology overview}
    \label{fig:overview-methods}
\end{figure}

\subsection{Overview}
\label{sec:model-overview}
\input{../renewable-energy-mdo/sections/model-overview.tex}







