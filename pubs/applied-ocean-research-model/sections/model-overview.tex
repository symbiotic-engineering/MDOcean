\section{Modeling Methodology}
\label{sec:model}

This paper places considerable emphasis not only on the specific methods used but also on the process to select these methods, informed through a lens of MDO and systems thinking, with the intent to guide readers interested in conducting their own multidisciplinary WEC optimization.
In section~\ref{sec:model-overview}, the research objective and design scope are first used to define a broad problem formulation and module decomposition.
Next, section~\ref{sec:modules} describes the construction of the simulation model.
It develops the assumptions, analysis method, and implementation details of each module based on the appropriate balance of accuracy and speed.
Sections \ref{sec:validation} and \ref{sec:sim-runtime} discuss model validation and runtime benchmarking.
The simulation capabilities and limitations are then combined with the original broad design optimization intent to inform a detailed optimization problem formulation.
In this step, the exact objectives, design variables, constraints, and parameters are selected and outlined in section~\ref{sec:formulation}.
The resulting optimization problem is then characterized and its properties ultimately inform the solution method including algorithm selection, discussed in section~\ref{sec:optim-process}.
Finally, the process is tweaked iteratively until it yields satisfactory results, with noteworthy insights integrated into the relevant sections \ref{sec:model-overview} to \ref{sec:optim-process}.
Figure~\ref{fig:overview-methods} depicts this methodology visually.
\begin{figure}
    \centering
    \includegraphics[width=1\linewidth]{\matlabFilepath{2}}
    \caption{Methodology overview}
    \label{fig:overview-methods}
\end{figure}

\subsection{Overview}
\label{sec:model-overview}
The simulation is broken up into five modules: geometry, hydrodynamics, dynamics and control, structures, and economics.
The geometry module uses both bulk dimensions and structural thicknesses to calculate relevant areas, volumes, masses, measures of center, hydrostatics, and stability margins.
Hydrodynamics uses the bulk dimensions to calculate hydrodynamic coefficients using a semi-analytical method.
The dynamics and control module uses the mass, hydrodynamic coefficients, and generator ratings to determine the loads, response amplitude, and power production using a linear frequency domain model adapted to handle specific nonlinearities including drag and powertrain saturation behavior.
It considers both operational and storm design load cases.
The structures module takes in these loads along with bulk dimensions and structural thicknesses to calculate the stress and factor of safety to various failure criterion.
Finally, the economics module calculates the PTO and structural capital costs from generator ratings and material usage respectively, and combines it with the power matrix to estimate the levelized cost of energy.

The extended design structure matrix (XDSM) diagram in Figure \ref{fig:n2} illustrates the functional interfaces between modules.
XDSM diagrams are standard in MDO, with more details available in \cite{lambe_extensions_2012}.
Variables above and below each module are inputs, and variables to the left and right are outputs.
Design variables are shown in the first row.
The optimizer uses the objective $J$ and constraint $g$ outputs from a simulation run to inform the next iteration of design variables $x$, iterating until ultimately converging to a set of optimal values $x^*$, $J^*$, and $g^*$.
The lack of variables in the lower left portion of the diagram indicates that there is no feedback coupling between modules, so an external solver to enforce consistency is not necessary.
This helps decrease the simulation computation time.
The presence of variables in the upper right of the diagram indicates feed-forward coupling in which the output of one module directly affects subsequent modules.
\begin{figure}
\centering
\includegraphics[width=\linewidth]{\matlabFilepath{3}}
\caption{Simplified $N^2$/XDSM diagram}\label{fig:n2}
\end{figure}
When coupling is \textit{non-monotonic}, concurrent optimization is required to obtain the system optimal design, and optimizing each module sequentially or in parallel would result in a sub-optimal system design.
The WEC coupling here is non-monotonic because perturbing a variable from one module in a certain direction does not necessarily determine the direction of the propagated change in coupling variables, objective, and constraints computed in other modules.
With an objective $J$ of LCOE, for example, while an increase in hydrodynamic damping generally increases power production (beneficial to $J$), it also increases structural loading (detrimental to $g$), and the bulk dimensions required to achieve that higher damping may produce higher stresses (detrimental to $g$) and increase the required structural material (detrimental to $J$).
Thus, it would be sub-optimal to optimize the hydrodynamics module strictly for damping or power production followed by a separate optimization considering structures and economics.
Furthermore, while even in a unified optimization it may be tempting to calculate the structural thicknesses within the structures module as the minimum required to sustain loads without failure, it is important to keep structural thicknesses as design variables because they also contribute to the hydrostatic constraints computed in the geometry module.
If active, these constraints could make the system-level optimum material thickness larger than what is structurally necessary.
Reference~\cite{papalambros_principles_2017} describes the monotonicity checking procedure more fully.
Using a monolithic MDO architecture, in which a single optimizer drives the design of all modules and considers all constraints, addresses this non-monotonic coupling. 

All modules except the dynamics module are explicit, meaning they require no internal iteration to converge.
The dynamics module requires iteration to incorporate nonlinearities into the quasi-linear frequency domain model, shown in the XDSM diagram as the orange box.
The dynamics iteration occurs separately for each outer iteration of the optimizer.
This structure is known as the multiple discipline feasible (MDF) architecture.
In principle, it is possible to remove the dynamics iteration and instead incorporate the nonlinear dynamics as a residual equality constraint within the optimization, which is known as the simultaneous analysis and design (SAND) architecture \cite{martins_multidisciplinary_2013}.
SAND generally decreases the runtime of the simulation (analysis) but requires more optimization iterations, since it is the optimization rather than the simulation that must converge residuals. 
The WecOptTool control co-optimization software pursues the SAND strategy by collocating the dynamics constraints with the pseudo-spectral method \cite{coe_initial_2020}.

Readers familiar with trajectory optimization should note that the choice of MDF versus SAND as an MDO architecture parallels that of shooting versus collocation as a transcription method, in the sense that one chooses to enforce governing equations with simulation versus optimization \cite{underactuated}.
However, in MDO the appropriate decision depends on characteristics of all modules, not merely the module whose governing equations are in question.
As \sectionautorefname~\ref{sec:sim-runtime} will demonstrate, the hydrodynamics module, not dynamics and controls, is the dominant computational cost of the simulation.
This means that even a substantial dynamics speedup represents only a small speedup of the full simulation and is unlikely to outweigh the slowdown of more (hydrodynamics) simulation evaluations.
For this reason, SAND would likely increase the runtime of the full optimization, motivating MDOcean's selection of MDF as the more suitable architecture.
Another advantage of MDF over SAND is that the latter invites the possibility of inconsistent dynamics if the optimization terminates unsuccessfully \cite{martins_multidisciplinary_2013} and requires optimizing a dummy constant objective to perform simulation without optimization.


