\section{Insights and Discussion}
\label{sec:discussion}
\subsection{Variable Sweeps}
\subsubsection{Hydrodynamic Efficiency}
(show MEEM sweep and power/volume pareto?)
A full factorial sweep obtaining cw/cwmax and volume across
m0h
a1/a2
m0a2
d1/h
d2/d1

Nondimensionalize power by wave power and volume like I did in umerc paper

\subsubsection{Damping Plate Size}
In the damping plate, stress can be reduced substantially by increasing the plate inner radius, because the same force has a lower lever arm to the column and therefore creates less bending moment.
\begin{figure}
    \centering
    \includegraphics[width=0.75\linewidth]{\matlabFilepath{41}}
    \caption{Effect of damping plate aspect ratio on maximum stress and deflection}
    \label{fig:damping-plate-maxs}
\end{figure}

\subsubsection{Effect of PTO Force Limit}
Previous work \hl{cite} has shown that capping the PTO force can substantially reduce PTO size and cost with minimal decrease in power.
Reference \cite{mccabe_force-limited_2024} shows that for a given sea state, power decreases quadratically with decreasing force limit for the worst-case scenario of zero intrinsic reactance, and that highly reactive devices are even less sensitive to force limits.
Intuitively, the sensitivity of annual average power to force limit should be even lower when considering a range of realistic sea states, because large sea states that require higher forces are comparatively rare.
Using MDOcean, we investigate the effect of PTO force limit on average power and structural load in \figureautorefname~\ref{fig:force-power-limit}.

\begin{figure}
\centering
\includegraphics[width=.8\linewidth]{\matlabFilepath{35}}
\caption{Effect of Force and Power Limit}
\label{fig:force-power-limit}
\end{figure}

Decreasing the force limit has nearly no effect until the force limit is around 50\% of the nominal value, after which power falls off steeply.
On the other hand, after a brief region of insensitivity, the structural load for the operational design load case scales nearly linearly with the force limit.
The insensitive region exists because at high force limits, the hydrodynamic force rather than the PTO force dominates the overall load, although at force limits below a threshold, PTO load dominates.

\subsection{Multidisciplinary Insights}
This section leverages the analytical multidisciplinary nature of the model to draw intuitive insights on limit cases and tradeoffs, and observe nondimensional relationships and scaling laws that would not be readily apparent in a purely numerical or single-discipline model.

- Formulation of power matrix as a multiplication of drag, linear shape, and PTO efficiency matrices

- Design of experiments

- Damping vs reactive control: under what price conditions does the extra power of reactive control justify the extra cost compared to damping control?



\subsection{Future Work}